\section{Oglašavanje}
Oglašavanje predstavlja vid saradnje između časopisa i zakupca oglasnog prostora. To je mogućnost za zakupca da putem oglasa u časopisu promoviše svoj proizvod ili uslugu, dok sa druge strane predstavlja materijalnu dobit za časopis.

Imajući u vidu obostranu korist, oglasni prostor je skoro uvek obavezan u sastavu jednog časopisa.

\subsection{Određivanje cene oglasnog prostora}
\begin{description}
\item [Opis] Podela oglasnog prostora prema cenovnom rangu.
\item [Učesnici] Menadžment.
\item [Preduslov] Određene pozicije oglasa u časopisu.
\item [Postuslov] Oglasni prostor kategorisan prema ceni.
\end{description}
\subsubsection{Glavni tok}
Menadžment zajedno sa glavnim urednikom analizira postojeće sekcije u časopisu. Prema njihovoj aktuelnosti, određuje cenu prostora za oglas koji pripada sekciji.
Cena može biti korigovana u budućnosti u zavisnosti od ponude i potražnje.

\subsection{Zakupljivanje oglasnog prostora}
\begin{description}
\item [Opis] Postizanje dogovora između menadžmenta i zakupca oglasnog prostora.
\item [Učesnici] Zakupac oglasa, menadžment.
\item [Preduslov] Definisan prostor u časopisu predviđen za oglae kao i njegova cena
\item [Postuslov] Uspešno prodat oglasni prostor.
\end{description}
\subsubsection{Glavni tok}
Zakupac kontaktira menadžment kako bi se informisao o cenama oglasa. Menadžment saopštava cene oglasnog prostora po kategorijama.
Zakupac shodno svojim željama i finansijkim mogućnostima pravi izbor, saopštava menadžmentu onaj prostor koji je voljan da zakupi. Potom šalje oglase menadžmentu koje želi da objavi. Eventualno se pretplaćuje na oglasni prostor na duži vremenski period.
Menadžment i zakupac formalno sklapaju ugovor o zakupljenom oglasnom prostoru.
