\chapter{Uvod}

 Ovaj rad se bavi modeliranjem informacionog sistema Izdavaštva jednog časopisa. Modelira se deo sistema koji se odnosi na stvaranje sadržaja časopisa i njegovo ob\-ja\-vlji\-va\-nje. Razmatrani časopis ima pisano kao i Internet izdanje. Radi sticanja potrebnog ra\-zu\-me\-va\-nja za analizu ovog sistema izvršeno je skupljanje i razmatranje informacija sa Interneta. Rezultat je osnovna ideja o entitetima i njihovim odnosima u sistemu kao i procesima koji se u njemu odvijaju. \\

 U pogledu poboljšanja, predložili bi smo jasnije razgraničenje dužnosti izvršnih urednika pomoću strožijeg određivanja rubrika i jasnijeg praćenja koji članak kojoj pripada i da li je već pregledan. U ovome bi pomoglo i uvođenje digitalnog sistema za praćenje revizija koji bi omogućio jasan uvid u tražene i izvršene promene nad sastavnim delovima sadržaja časopisa.\\


Rad je izrađen kao grupni studentski projekat na Matematičkom fakultetu, na stu\-dij\-skom programu Informatika, prve godine Master studija. Projekat je odrađen pod nadzorom profesora dr Saše Malkova, u okviru predmeta Informacioni sistemi. \\

\section*{Učesnici u sistemu}

\noindent U okviru izdavaštva postoje dve organizacione celine:
\begin{enumerate}
\item Menadžment (administracija)
\item Uredništvo i pisci
\end{enumerate}

\subsection*{Menadžment}

Menadžment se bavi poslovnom stranom vođenja časopisa: sklapanjem ugovora, is\-pla\-tom plata, računovodstvom, nabavkom sredstava i odabirom saradnika (u dogovoru sa uredništvom).
Menadžment nije fokus ovog rada, ali učestvuje u poslovima koji se tiču stvaranja sadržaja. Menadžment odgovara izvršnom odboru koji opet odgovara vlasnicima časopisa. Oni su sasvim van obima ovog opisa.

\subsection*{Uredništvo i pisci}

Uredništvo i oni sa kojima rade: pisci, dizajneri, fotografi i drugi direktno i indirektno stvaraju sadržaj. Sve što se proizvede mora biti provereno i odobreno od strane nekog urednika. 

Među urednike spadaju:
\begin{itemize}
\item Glavni odgovorni urednik
\item Izvršni urednici (pisanog i Internet izdanja)
\item Grafički urednik
\item Lektori
\end{itemize}

Glavni odgovorni urednik je zadužen za postavljanje drugih urednika i mora odobriti direktno ili indirektno sav sadržaj koji časopis objavljuje. On odlučuje o rubrikama koje časopis sadrži i o duhu časopisa uopšte. Njega takođe postavlja izvršni odbor. \\

Izvršni urednici su zaduženi za određene rubrike, oni predlažu piscima teme i uređuju njihov  rad tako da se uklopi u njihovu kreativnu viziju. Svaki članak u  njihovoj oblasti mora biti pročitan od strane njih i odobren. Urednici koji rade na Internet izdanju moraju se pored objavljenih članaka starati i o samoj veb stranici koju održava administrator i o uređivanju komentara koji na sajt pristižu, za šta zadužuju moderatore. \\

Grafički urednik je odgovoran za izgled časopisa, u štampanoj kao i onlajn verziji. Sa njim rade dizajneri, fotografi i ilustratori. On je takođe u dogovoru sa menadžmentom odgovoran za licenciranje slika koje ne pripadaju časopisu. \\

Lektori proveravaju svaki članak, uklanjaju pravopisne i gramatičke greške, staraju se o tome da članci odgovaraju stilskim standardima časopisa i proveravaju činjenice navedene u tekstu. \\

Ostali zaposleni su:
\begin{itemize}
\item  Pisci, odnosno kolumnisti, izveštači, novinari, intervjuisti i drugi koji proizvode članke bilo kao stalni zaposleni ili spoljni saradnici i predaju ih redakciji.
\item  Dizajneri, fotografi i ilustratori kreiraju vizuelni sadržaj za grafičkog urednika, uljučujući tipografiju i boje stranica.
\item  Administratori i moderatori koji održavaju veb stranicu i odgovaraju Internet redakciji.
\end{itemize}

\section*{Opis rada izdavaštva}

Da bi pisac mogao obavljati svoj posao, potrebno je da bude dodeljen nekom uredniku koji će odgovarati za njegov rad i kome predaje članke. Urednik, čija oblast može biti štampana ili Internet rubrika, prima članak i dorađuje ga i eventualno ga šalje nazad piscu na doradu. Kada su i pisac i urednik zadovoljni člankom, urednik ga šalje lektoru koji ispravlja greške i pravi manje izmene. \\

Zatim grafički urednik sa svojim saradnicima ilustruje i bira izgled članka koji se zatim u slučaju Internet izdanja predaje administratoru za objaviti a u slučaju štampanog uklapa sa ostalim člancima i predaje glavnom odgovornom direktoru koji daje svoje odobrenje pre predaje štampariji. Časopis se tada prodaje distributerima i šalje pret\-plat\-ni\-ci\-ma. \\

Kao dodatni izvor zarade, časopis ima i oglase, a da bi oglašavač mogao dobiti svoj oglas, on mora postići dogovor sa menadžmentom kao i sa uredništvom. Kada je to postignuto, oglašavač dostavlja materijal svog oglasa (obično sliku) a grafički urednik se stara o tome gde će on stajati i po potrebi pravi manje vizuelne izmene. \\
