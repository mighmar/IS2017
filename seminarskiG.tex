\documentclass[a4paper,12pt]{extarticle}
\usepackage{fancyhdr}
\pagestyle{fancy}
\usepackage{amsfonts}
\usepackage{amsmath}
\usepackage{tikz}
\usepackage[utf8]{inputenc}
\usepackage[serbianc]{babel}
\usepackage{titling}
\usepackage{caption}
\usepackage{subcaption}




\usepackage[margin=2.5cm]{geometry}
\addtolength{\hoffset}{0.5cm}
\addtolength{\textwidth}{-0.5cm}





\begin{document}
\renewcommand*\contentsname{ }
\begin{titlepage}
\begin{center}

	\textbf{\Large Univerzitet u Beogradu}\\[0.5cm]
	
	{\large Matematički fakultet}\\[6cm]
	
	\textsc{\huge Grupni studentski rad}\\[0.5cm]
	
	\textbf{\Large Informacioni sistem izdavaštva časopisa}\\[10cm]

	\begin{tabular}{|l|l|}
  		\hline
	    Studenti: & Ozren Demonja \\ \cline{2-2}
	              & Stefan Maksimović \\ \cline{2-2}
	              & Marko Crnobrnja \\ \hline
	    Predmet: & Informacioni sistemi (R271) \\ \hline
	    Školska godina: & 2016/2017. \\ \hline
	    Profesor: & Dr Saša Malkov	 \\ \hline
	    Datum &  \\ \hline
	\end{tabular}

\end{center}
\end{titlepage}

\newpage \section{Uvod}

Ovaj rad se bavi modeliranjem informacionog sistema Izdavaštva jednog časopisa. Modelira se deo sistema koji se odnosi na stvaranje sadržaja časopisa i njegovo objavljivanje. Razmatrani časopis ima pisano kao i Internet izdanje. Radi sticanja potrebnog razumevanja za analizu ovog sistema izvršeno je skupljanje i razmatranje informacija sa Interneta. Rezultat je osnovna ideja o entitetima i njihovim odnosima u sistemu kao i procesima koji se u njemu odvijaju. \\

 U pogledu poboljšanja, predložili bi smo jasnije razgraničenje dužnosti izvršnih urednika pomoću strožijeg određivanja rubrika i jasnijeg praćenja koji članak kojoj pripada i da li je već pregledan. U ovome bi pomoglo i uvođenje digitalnog sistema za praćenje revizija koji bi omogućio jasan uvid u tražene i izvršene promene nad sastavnim delovima sadržaja časopisa.\\


Rad je izrađen kao grupni studentski projekat na Matematičkom fakultetu, na studijskom
programu Informatika, prve godine Master studija. Projekat je odrađen pod nadzorom profesora dr Saše Malkova, u okviru predmeta Informacioni sistemi. \\

\subsection{Učesnici u sistemu}

U okviru izdavaštva postoje dve organizacione celine:
\begin{enumerate}
\item Menadžment (administracija)
\item Uredništvo i pisci
\end{enumerate}

\subsubsection{Menadžment}

Menadžment se bavi poslovnom stranom vođenja časopisa: sklapanjem ugovora, isplatom plata, računovodstvom, nabavkom sredstava i odabirom saradnika (u dogovoru sa uredništvom).
Menadžment nije fokus ovog rada, ali učestvuje u poslovima koji se tiču stvaranja sadržaja. Menadžment odgovara izvršnom odboru koji opet odgovara vlasnicima časopisa. Oni su sasvim van obima ovog opisa.

\subsubsection{Uredništvo i pisci}

Uredništvo i oni sa kojima rade: pisci, dizajneri, fotografi i drugi direktno i indirektno stvaraju sadržaj. Sve što se proizvede mora biti provereno i odobreno od strane nekog urednika. 

Među urednike spadaju:
\begin{itemize}
\item Glavni odgovorni urednik
\item Izvršni urednici (pisanog i Internet izdanja)
\item Grafički urednik
\item Lektori
\end{itemize}

Glavni odgovorni urednik je zadužen za postavljanje drugih urednika i mora odobriti direktno ili indirektno sav sadržaj koji časopis objavljuje. On odlučuje o rubrikama koje časopis sadrži i o ***smeru*** časopisa uopšte. Njega takođe postavlja izvršni odbor. \\

Izvršni urednici su zaduženi za određene rubrike, oni predlažu piscima teme i uređuju njihov  rad tako da se uklopi u njihovu kreativnu viziju. Svaki članak u  njihovoj oblasti mora biti pročitan od strane njih i odobren. Urednici koji rade na Internet izdanju moraju se pored objavljenih članaka starati i o samoj veb stranici koju održava administrator i o uređivanju komentara koji na sajt pristižu, za šta zadužuju moderatore. \\

Grafički urednik je odgovoran za izgled časopisa, u štampanoj kao i onlajn verziji. Sa njim rade dizajneri, fotografi i ilustratori. On je takođe u dogovoru sa menadžmentom odgovoran za licenciranje slika koje ne pripadaju časopisu. \\

Lektori proveravaju svaki članak, uklanjaju pravopisne i gramatičke greške, staraju se o tome da članci odgovaraju stilskim standardima časopisa i proveravaju činjenice navedene u tekstu. \\

Ostali zaposleni su:
\begin{itemize}
\item  Pisci, odnosno kolumnisti, izveštači, novinari, intervjuisti i drugi koji proizvode članke bilo kao stalni zaposleni ili spoljni saradnici i predaju ih redakciji.
\item  Dizajneri, fotografi i ilustratori kreiraju vizuelni sadržaj za grafičkog urednika, uljučujući tipografiju i boje stranica.
\item  Administratori i moderatori koji održavaju veb stranicu i odgovaraju Internet redakciji.
\end{itemize}

\subsection{Opis rada izdavaštva}

Da bi pisac mogao obavljati svoj posao, potrebno je da bude dodeljen nekom uredniku koji će odgovarati za njegov rad i kome predaje članke. Urednik, čija oblast može biti štampana ili Internet rubrika, prima članak i dorađuje ga i eventualno ga šalje nazad piscu na doradu. Kada su i pisac i urednik zadovoljni člankom, urednik ga šalje lektoru koji ispravlja greške i pravi manje izmene. \\

Zatim grafički urednik sa svojim saradnicima ilustruje i bira izgled članka koji se zatim u slučaju Internet izdanja predaje administratoru za objaviti a u slučaju štampanog uklapa sa ostalim člancima i predaje glavnom odgovornom direktoru koji daje svoje odobrenje pre predaje štampariji. Časopis se tada prodaje distributerima i šalje pretplatnicima. \\

Kao dodatni izvor zarade, časopis ima i oglase, a da bi oglašavač mogao dobiti svoj oglas, on mora postići dogovor sa menadžmentom kao i sa uredništvom. Kada je to postignuto, oglašavač dostavlja materijal svog oglasa (obično sliku) a grafički urednik se stara o tome gde će on stajati i po potrebi pravi manje vizuelne izmene. \\







\newpage 
\begin{center}
	\textbf{\large SADRŽAJ}
\end{center}
\tableofcontents 

\newpage


\end{document}
